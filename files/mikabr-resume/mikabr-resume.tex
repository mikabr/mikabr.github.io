\documentclass[11pt,]{article}
\usepackage{lmodern}
\usepackage{amssymb,amsmath}
\usepackage{ifxetex,ifluatex}
\usepackage{fixltx2e} % provides \textsubscript
\ifnum 0\ifxetex 1\fi\ifluatex 1\fi=0 % if pdftex
  \usepackage[T1]{fontenc}
  \usepackage[utf8]{inputenc}
\else % if luatex or xelatex
  \ifxetex
    \usepackage{mathspec}
  \else
    \usepackage{fontspec}
  \fi
  \defaultfontfeatures{Ligatures=TeX,Scale=MatchLowercase}
    \setmainfont[]{SourceSansPro-Regular}
    \setsansfont[]{SourceSansPro-Regular}
    \setmonofont[Mapping=tex-ansi]{SourceCodePro-Regular}
\fi
% use upquote if available, for straight quotes in verbatim environments
\IfFileExists{upquote.sty}{\usepackage{upquote}}{}
% use microtype if available
\IfFileExists{microtype.sty}{%
\usepackage{microtype}
\UseMicrotypeSet[protrusion]{basicmath} % disable protrusion for tt fonts
}{}
\usepackage[margin=.8in]{geometry}


\usepackage{graphicx,grffile}
\makeatletter
\def\maxwidth{\ifdim\Gin@nat@width>\linewidth\linewidth\else\Gin@nat@width\fi}
\def\maxheight{\ifdim\Gin@nat@height>\textheight\textheight\else\Gin@nat@height\fi}
\makeatother
% Scale images if necessary, so that they will not overflow the page
% margins by default, and it is still possible to overwrite the defaults
% using explicit options in \includegraphics[width, height, ...]{}
\setkeys{Gin}{width=\maxwidth,height=\maxheight,keepaspectratio}

% % 
\setlength{\emergencystretch}{3em}  % prevent overfull lines
\providecommand{\tightlist}{%
  \setlength{\itemsep}{0pt}\setlength{\parskip}{0pt}}
\setcounter{secnumdepth}{0}
% Redefines (sub)paragraphs to behave more like sections
\ifx\paragraph\undefined\else
\let\oldparagraph\paragraph
\renewcommand{\paragraph}[1]{\oldparagraph{#1}\mbox{}}
\fi
\ifx\subparagraph\undefined\else
\let\oldsubparagraph\subparagraph
\renewcommand{\subparagraph}[1]{\oldsubparagraph{#1}\mbox{}}
\fi

% Now begins the stuff that I added.
% ----------------------------------



% Custom section fonts
\usepackage{sectsty}
%\sectionfont{\rmfamily\mdseries\large\bf}
%\subsectionfont{\rmfamily\mdseries\normalsize\itshape}


% Make lists without bullets
%\renewenvironment{itemize}{
%  \begin{list}{}{
%    \setlength{\leftmargin}{1.5em}
%  }
%}{
%  \end{list}
%}


% Make parskips rather than indent with lists.
\usepackage{parskip}
\usepackage{titlesec}

% \setlength{\parindent}{15pt}

\titlespacing\section{0pt}{12pt plus 4pt minus 2pt}{4pt plus 2pt minus 2pt}
\titlespacing\subsection{0pt}{12pt plus 4pt minus 2pt}{4pt plus 2pt minus 2pt}

\titleformat{\section}
  {\rmfamily\mdseries\Large\bf\sc}{\thesection}{1em}{}[{\titlerule[0.4pt]}\vspace{2pt}]

% Use fontawesome. Note: you'll need TeXLive 2015. Update.
\defaultfontfeatures{
    Path = /usr/local/texlive/2019/texmf-dist/fonts/opentype/public/fontawesome/ }
\usepackage{fontawesome}

% Fancyhdr, as I tend to do with these personal documents.
\usepackage{fancyhdr,lastpage}
\pagestyle{fancy}
\renewcommand{\headrulewidth}{0.0pt}
\renewcommand{\footrulewidth}{0.0pt}
\lhead{}
\chead{}
\rhead{}
\lfoot{}
\cfoot{}
\rfoot{}

\usepackage{enumitem}
\setlist[itemize]{leftmargin=*,label=—}

\newcommand{\project}[4]{
  \textbf{#1} \hspace{4pt} \small \href{http://#2}{#2} \normalsize \hfill #3\\ \hangindent=15pt #4
}

% \newcommand{\iconPad}{\hspace{.5mm}}
% \newcommand{\iconSep}{\hspace{2mm}}
\newcommand{\iconPad}{.25mm}
\newcommand{\iconSep}{2.5mm}

% Always load hyperref last.
\usepackage{hyperref}

\hypersetup{unicode=true,
            pdftitle={Mika Braginsky:  Resume},
            pdfauthor={Mika Braginsky},
            colorlinks=true,
            linkcolor=Maroon,
            citecolor=Blue,
            % urlcolor=Blue,
            urlcolor=[RGB]{51,34,136},
            breaklinks=true, bookmarks=true}
\urlstyle{same}  % don't use monospace font for urls

\begin{document}

\centerline{\Huge Mika Braginsky}
% \begin{centering}

% {\huge Mika Braginsky}

\vspace{1.5 mm}

% \noindent\hfil\rule{0.5\textwidth}{.4pt}\hfil
% \hrule
% \rule{0.5\textwidth}{.4pt}

% \vspace{2 mm}

% \small


\moveleft.5\hoffset\centerline{ \scalebox{1}{\faCommentO}\hspace{\iconPad} they/them \hspace{\iconSep}
\scalebox{1}{\faEnvelopeO}\hspace{\iconPad} \href{mailto:mika.br@gmail.com}{\nolinkurl{mika.br@gmail.com}} \hspace{\iconSep}
\scalebox{1.1}{\faGithub}\hspace{\iconPad} \href{http://github.com/mikabr}{mikabr} \hspace{\iconSep} 
\scalebox{1}{\faExternalLink}\hspace{\iconPad} \href{http://mikabr.io}{mikabr.io}  \hspace{\iconSep} 
} %   }
% \moveleft.5\hoffset\centerline{Pronouns: they/them}
% \normalsize

% \vspace{2 mm}

% \hrule
% \rule{0.5\textwidth}{.4pt}

% \end{centering}

\hypertarget{education}{%
\section{\texorpdfstring{\faGraduationCap \hspace{1pt}
Education}{ Education}}\label{education}}

\textbf{Massachusetts Institute of Technology} \hfill Sep 2016 --
Present\\
\hspace*{0.333em}\hspace*{0.333em}\hspace*{0.333em}\hspace*{0.333em}Ph.D.~Candidate,
Brain and Cognitive Sciences\\
\hspace*{0.333em}\hspace*{0.333em}\hspace*{0.333em}\hspace*{0.333em}NSF
Graduate Research Fellowship Program Fellow

\textbf{Massachusetts Institute of Technology} \hfill Sep 2010 -- Jun
2014\\
\hspace*{0.333em}\hspace*{0.333em}\hspace*{0.333em}\hspace*{0.333em}B.S.
in Computer Science and Engineering \& Brain and Cognitive Sciences

\hypertarget{open-source-software}{%
\section{\texorpdfstring{\scalebox{1.3}{\faCodeFork} \hspace{1.5pt} Open
source software}{  Open source software}}\label{open-source-software}}

\emph{This list is a sample of some of the R-based open source software
projects for which I am or was the primary developer.}

\project{peekbankr}{github.com/langcog/peekbankr}{Aug 2019 -- Present}{R package for accessing peekbank, a database of developmental eye-tracking datasets.}

\project{jglmm}{github.com/mikabr/jglmm}{Dec 2018 -- Present}{R package for fitting generalized linear mixed effects models in Julia.}

\project{ggpirate}{github.com/mikabr/ggpirate}{Sep 2017 -- Present}{R package for making more informative alternatives to bar plots.}

\project{rwebppl}{github.com/mhtess/rwebppl}{Feb 2016 -- Present}{R package providing an interface to the WebPPL probabilistic programming language.}

\project{wordbankr}{github.com/langcog/wordbankr}{Jul 2015 -- Present}{R package for accessing Wordbank, a database of vocabulary development data.}

\project{tidyboot}{github.com/langcog/tidyboot}{Jun 2015 -- Present}{R package for doing tidyverse-compatible bootstrapping.}

\project{Wordbank}{wordbank.stanford.edu}{Mar 2014 -- Aug 2017}{Database of vocabulary development data (using Python and MySQL).\\ Interactive tools for its access (using R Shiny).\\ Demo presented at RStudio Shiny Developer Conference 2016.}

\hypertarget{teaching-experience}{%
\section{\texorpdfstring{\faUsers \hspace{2pt} Teaching
experience}{ Teaching experience}}\label{teaching-experience}}

\project{Teaching Assistant, Laboratory in Psycholinguistics}{web.mit.edu/psycholinglab/public}{2018 -- Present}{Develop and teach weekly labs on tidyverse-based data analysis and visualization.\\Received two departmental awards for Excellence in Undergraduate Teaching (2019, 2020).}

\project{Workshop Instructor, LSA Linguistic Institute 2019}{mikabr.io/acq-tools/acq-tools-lsa.html}{July 2019}{Created and taught workshop on the use of large-scale data tools in child language acquisition.}

\hypertarget{other-experience}{%
\section{\texorpdfstring{\faArchive \hspace{2pt} Other
experience}{ Other experience}}\label{other-experience}}

\project{Co-Author, \em{Variability and Consistency in Early Language Learning:
The Wordbank Project}}{}{MIT Press, March 2021}{\small \href{http://wordbank-book.stanford.edu}{wordbank-book.stanford.edu} \normalsize\\Co-authored a book of data-driven analyses of children's language learning.}

\project{Independent Contractor, rOpenSci}{}{Jun 2016 -- Jul 2016}{Developed an R package for automatic deployment of R package documentation by Travis CI.}

\end{document}
